\chapter{Thực nghiệm Sư phạm}

\section{Mục tiêu thực nghiệm}\begin{itemize}
	\item Tìm hiểu khả năng ứng dụng AI vào đánh giá năng lực của học sinh.
	\item Phân tích sự chính xác của kết quả kiểm tra thích ứng so với điểm trung bình môn Toán của học sinh, từ đó nhận xét về tính khả quan của Kant bot trong đánh giá năng lực học sinh.
	\item Cải tiến, hoàn thiện Kant và công bố API dưới dạng \textit{mã nguồn mở} (open-source).
\end{itemize}

\section{Đối tượng và tiến trình thực nghiệm}
\subsection{Đối tượng thực nghiệm}\begin{itemize}
	% \item Giáo viên bộ môn Toán trường THPT thực hành Sư phạm.
	\item 38 học sinh lớp 11, 12 thuộc các trường THPT Châu Văn Liêm, Lý Tự Trọng và Thực hành Sư phạm.
\end{itemize}

\subsection{Tiến trình thực nghiệm}
\begin{enumerate}[label=\textbf{Giai đoạn \arabic*.},align=left,left=0cm..0cm,itemindent=*]
	\item Gửi đường link truy cập cho học sinh làm thực nghiệm phần Tổ hợp – Xác suất với sự đánh giá tự động của Kant bot.\par
	\item Tổng hợp kết quả đánh giá của học sinh.
	\item Phân tích quá trình đánh giá của một số học sinh.
\end{enumerate}

\section{Nội dung thực nghiệm}
Quá trình thực nghiệm chính được thực hiện hoàn toàn trực tuyến với Kant bot, thông qua nền tảng Facebook Messenger.\par

\subsection{Thực nghiệm đánh giá học sinh}
Trong phần này, người dùng cần phải thực hiện tối đa 10 câu hỏi trắc nghiệm cho một chủ đề – với câu hỏi được đưa ra dựa trên dữ liệu trả lời những câu hỏi trước – để đạt tới mức kỳ vọng ổn định nhất. Sau đó Kant sẽ đưa ra kết quả đánh giá năng lực. Hình \ref{fig:fig-c4-chatbot-demo} minh họa quá trình thực nghiệm đánh giá với Kant.
\begin{figure}[htb!]\centering
	\includegraphics[width=7cm]{kant-ux/chat-1}
	\includegraphics[width=7cm]{kant-ux/chat-2}
	\includegraphics[width=7cm]{kant-ux/chat-3}
	\includegraphics[width=7cm]{kant-ux/chat-4}
	\caption{Minh họa quá trình thực nghiệm với Kant}
	\label{fig:fig-c4-chatbot-demo}
\end{figure}\par

% \subsection{Bảng khảo sát giáo viên}
% \begin{enumerate}[label=\textbf{M\arabic*.},align=left,left=0cm..0cm,itemindent=*]
% 	\item Thầy/cô cho rằng chatbot hỗ trợ cho học sinh tự học môn Toán hiệu quả?
% 		\begin{enumerate}[label=$\square$,align=left,left=1cm..0cm,itemindent=*]
% 			\item Hoàn toàn không đồng ý.
% 			\item Không đồng ý.
% 			\item Phân vân.
% 			\item Đồng ý.
% 			\item Hoàn toàn đồng ý.
% 		\end{enumerate}
% 	\item Thầy/cô cho rằng chatbot có thể giúp học sinh tự đánh giá năng lực Toán của bản thân?
% 		\begin{enumerate}[label=$\square$,align=left,left=1cm..0cm,itemindent=*]
% 			\item Hoàn toàn không đồng ý.
% 			\item Không đồng ý.
% 			\item Phân vân.
% 			\item Đồng ý.
% 			\item Hoàn toàn đồng ý.
% 		\end{enumerate}
% 	\item Thầy/cô cho rằng chatbot có thể góp phần nâng cao năng lực Toán học của học sinh?
% 		\begin{enumerate}[label=$\square$,align=left,left=1cm..0cm,itemindent=*]
% 			\item Hoàn toàn không đồng ý.
% 			\item Không đồng ý.
% 			\item Phân vân.
% 			\item Đồng ý.
% 			\item Hoàn toàn đồng ý.
% 		\end{enumerate}
% 	\item Thầy/cô cho rằng chatbot trên nền tảng FB Messenger:
% 		\begin{enumerate}[label=\textbf{\alph*)},align=left,left=1cm..0cm,itemindent=*]
% 			\item Rất dễ tiếp cận?
% 				\begin{enumerate}[label=$\square$,align=left,left=2cm..0cm,itemindent=*]
% 					\item Hoàn toàn không đồng ý.
% 					\item Không đồng ý.
% 					\item Phân vân.
% 					\item Đồng ý.
% 					\item Hoàn toàn đồng ý.
% 				\end{enumerate}
% 			\item Có thể học bất cứ thời gian nào?
% 				\begin{enumerate}[label=$\square$,align=left,left=2cm..0cm,itemindent=*]
% 					\item Hoàn toàn không đồng ý.
% 					\item Không đồng ý.
% 					\item Phân vân.
% 					\item Đồng ý.
% 					\item Hoàn toàn đồng ý.
% 				\end{enumerate}
% 			\item Có thể học bất cứ nơi đâu có kết nối Internet?
% 				\begin{enumerate}[label=$\square$,align=left,left=2cm..0cm,itemindent=*]
% 					\item Hoàn toàn không đồng ý.
% 					\item Không đồng ý.
% 					\item Phân vân.
% 					\item Đồng ý.
% 					\item Hoàn toàn đồng ý.
% 				\end{enumerate}
% 			\item Cho kết quả phản hồi tức thì, không cần chờ đợi?
% 				\begin{enumerate}[label=$\square$,align=left,left=2cm..0cm,itemindent=*]
% 					\item Hoàn toàn không đồng ý.
% 					\item Không đồng ý.
% 					\item Phân vân.
% 					\item Đồng ý.
% 					\item Hoàn toàn đồng ý.
% 				\end{enumerate}
% 		\end{enumerate}
% 	\item Thầy/cô cho rằng chatbot là một công cụ hữu hiệu để góp phần nâng cao năng lực Toán học?
% 		\begin{enumerate}[label=$\square$,align=left,left=1cm..0cm,itemindent=*]
% 			\item Hoàn toàn không đồng ý.
% 			\item Không đồng ý.
% 			\item Phân vân.
% 			\item Đồng ý.
% 			\item Hoàn toàn đồng ý.
% 		\end{enumerate}
% 	\item Thầy/cô sẽ giới thiệu ứng dụng chatbot để học Toán cho bạn bè và học sinh của mình?
% 		\begin{enumerate}[label=$\square$,align=left,left=1cm..0cm,itemindent=*]
% 			\item Hoàn toàn không đồng ý.
% 			\item Không đồng ý.
% 			\item Phân vân.
% 			\item Đồng ý.
% 			\item Hoàn toàn đồng ý.
% 		\end{enumerate}
% 	\item Phí duy trì Chatbot hoạt động là 15 USD/tháng cho 500 người dùng là hợp lý (so với mức lương nhân sự, tốc độ và chất lượng phản hồi tin nhắn)?
% 		\begin{enumerate}[label=$\square$,align=left,left=1cm..0cm,itemindent=*]
% 			\item Hoàn toàn không đồng ý.
% 			\item Không đồng ý.
% 			\item Phân vân.
% 			\item Đồng ý.
% 			\item Hoàn toàn đồng ý.
% 		\end{enumerate}
% 	\item Thầy/cô nghĩ rằng để góp phần nâng cao năng lực Toán học, học sinh nên luyện tập với chatbot với tần suất nào?
% 		\begin{enumerate}[label=$\square$,align=left,left=1cm..0cm,itemindent=*]
% 			\item Hoàn toàn không đồng ý.
% 			\item Không đồng ý.
% 			\item Phân vân.
% 			\item Đồng ý.
% 			\item Hoàn toàn đồng ý.
% 		\end{enumerate}
% 	\item Theo thầy/cô, sử dụng chatbot trong dạy học Toán có ưu điểm gì?
% 		\begin{enumerate}[label=$\square$,align=left,left=1cm..0cm,itemindent=*]
% 			\item Hoàn toàn không đồng ý.
% 			\item Không đồng ý.
% 			\item Phân vân.
% 			\item Đồng ý.
% 			\item Hoàn toàn đồng ý.
% 		\end{enumerate}
% 	\item Theo thầy/cô, sử dụng chatbot trong dạy học Toán có nhược điểm gì?
% 		\begin{enumerate}[label=$\square$,align=left,left=1cm..0cm,itemindent=*]
% 			\item Hoàn toàn không đồng ý.
% 			\item Không đồng ý.
% 			\item Phân vân.
% 			\item Đồng ý.
% 			\item Hoàn toàn đồng ý.
% 		\end{enumerate}
% 	\item Thầy/cô có để xuất gì để cải thiện ứng dụng Chatbot được tốt hơn?
% 		\begin{enumerate}[label=$\square$,align=left,left=1cm..0cm,itemindent=*]
% 			\item Hoàn toàn không đồng ý.
% 			\item Không đồng ý.
% 			\item Phân vân.
% 			\item Đồng ý.
% 			\item Hoàn toàn đồng ý.
% 		\end{enumerate}
% 	\item Thầy/cô có thích Kant bot không?
% 		\begin{enumerate}[label=$\square$,align=left,left=1cm..0cm,itemindent=*]
% 			\item Hoàn toàn không thích.
% 			\item Không thích.
% 			\item Phân vân.
% 			\item Thích.
% 			\item Hoàn toàn thích.
% 		\end{enumerate}
% 	\item Trong thang điểm từ 1 đến 10, thầy/cô hãy cho điểm về cách thức hoạt động của Kant bot.
% 		\begin{multicols}{10}\begin{enumerate}[label=$\square$,align=left,left=0cm..0cm,itemindent=*]
% 			\item 1. \item 2. \item 3. \item 4. \item 5. \item 6. \item 7. \item 8. \item 9. \item 10.
% 		\end{enumerate}\end{multicols}
% \end{enumerate}

% \subsection{Phân tích tiên nghiệm bảng khảo sát giáo viên}
% Mẫu phiếu khảo sát được thiết kế với một số câu hỏi theo thang Linkert 5 bậc, cùng một số câu hỏi mở, và câu hỏi đánh giá, để nhằm thu thập thông tin nhiều và nhanh nhất
% \begin{enumerate}[label=\textbf{M\arabic*.},align=left,left=0cm..0cm,itemindent=*]
% 	\item Thầy/cô cho rằng chatbot hỗ trợ cho học sinh tự học môn Toán hiệu quả?\par
% 	Câu hỏi được thiết kế theo thang Likert 5 bậc để khảo sát ý kiến của giáo viên về tính khả quan của việc vận dụng chatbot vào tự học môn Toán ở học sinh.
% 	\item Thầy/cô cho rằng chatbot có thể giúp học sinh tự đánh giá năng lực Toán của bản thân?\par
% 	Câu hỏi được thiết kế theo thang Likert 5 bậc để khảo sát ý kiến của giáo viên về tính khả quan của việc học sinh tự đánh năng lực bản thân giá bằng công cụ chatbot.
% 	% \item Thầy/cô cho rằng chatbot có thể góp phần nâng cao năng lực Toán học của học sinh?
% 	\item Thầy/cô cho rằng chatbot trên nền tảng FB Messenger:
% 		\begin{enumerate}[label=\textbf{\alph*)},align=left,left=1cm..0cm,itemindent=*]
% 			\item Rất dễ tiếp cận?
% 			\item Có thể học bất cứ thời gian nào?
% 			\item Có thể học bất cứ nơi đâu có kết nối Internet?
% 			\item Cho kết quả phản hồi tức thì, không cần chờ đợi?
% 		\end{enumerate}
% 	% \item Thầy/cô cho rằng chatbot là một công cụ hữu hiệu để góp phần nâng cao năng lực Toán học?
% 	\item Thầy/cô sẽ giới thiệu ứng dụng chatbot để học Toán cho bạn bè và học sinh của mình?\par
% 	Câu hỏi được thiết kế theo thang Likert 5 bậc để khảo sát mong muốn giới thiệu chatbot hỗ trợ học tập Toán của giáo viên với bạn bè và học sinh. Qua đó phân tích được tiềm năng phát triển và phổ biến của chatbot trong giáo dục.
% 	% \item Phí duy trì Chatbot hoạt động là 15 USD/tháng cho 500 người dùng là hợp lý (so với mức lương nhân sự, tốc độ và chất lượng phản hồi tin nhắn)?
% 	\item Thầy/cô nghĩ rằng để góp phần nâng cao năng lực Toán học, học sinh nên luyện tập với chatbot với tần suất nào?\par
% 	Câu hỏi được thiết kế theo thang Likert 5 bậc để khảo sát ý kiến của giáo viên về mức độ thường xuyên trong việc học tập với chatbot của học sinh.
% 	\item Theo thầy/cô, sử dụng chatbot trong dạy học Toán có ưu điểm gì?\par
% 	Câu hỏi được thiết kế để phỏng vấn ý kiến của giáo viên về ưu điểm của chatbot trong học tập môn Toán phổ thông.
% 	\item Theo thầy/cô, sử dụng chatbot trong dạy học Toán có nhược điểm gì?\par
% 	Câu hỏi được thiết kế phỏng vấn ý kiến của giáo viên về ưu điểm của chatbot trong học tập môn Toán phổ thông.
% 	\item Thầy/cô có để xuất gì để cải thiện chatbot tốt hơn?\par
% 	Câu hỏi được thiết kế để khảo sát ý kiến góp ý của giáo viên khi vận dụng vào dạy học Toán, từ đó có thể cải thiện khả năng và tính khả quan của chatbot.
% 	\item Thầy/cô có thích Kant bot không?\par
% 	Câu hỏi được thiết kế để khảo sát thái độ của giáo viên đối với việc học tập với chatbot, cụ thể là Kant bot. Từ đó phân tích được tính khả thi khi áp dụng vào dạy học Toán học.
% 	\item Trong thang điểm từ 1 đến 10, thầy/cô hãy cho điểm về cách thức hoạt động của Kant bot.\par
% 	Câu hỏi được thiết kế để đánh giá mức độ hoàn thiện của Kant bot trong hỗ trợ dạy và học Toán ở bậc phổ thông.
% \end{enumerate}

\subsection{Kết quả thực nghiệm}

Có 38 học sinh hoàn thành phần kiểm tra năng lực chương Tổ hợp, xác suất. Kết quả đánh giá năng lực bằng Kant bot được tổng hợp ở bảng \ref{tab:tab-s4-result}, danh sách được sắp xếp và đặt mã số theo \textit{Sai số} giảm dần.

\begin{longtable}{ScSlScScScSc}
	\caption{Kết quả thực nghiệm đánh giá bằng Kant bot}\label{tab:tab-s4-result}\\
	\textbf{TT} & \multicolumn{1}{Sc}{\textbf{Trường THPT}} & \textbf{Lớp} & \textbf{KQ đánh giá} & \textbf{ĐTB Toán} & \textbf{Sai số} \\ \hline\endfirsthead

	\textbf{TT} & \multicolumn{1}{Sc}{\textbf{Trường THPT}} & \textbf{Lớp} & \textbf{KQ đánh giá} & \textbf{ĐTB Toán} & \textbf{Sai số} \\ \hline\endhead\hline\endfoot

	S01 & Thực hành Sư phạm & 11 & $8.5$ & $8.5$ & $0.0$ \\
	S02 & Thực hành Sư phạm & 11 & $9.6$ & $9.5$ & $0.1$ \\
	S03 & Châu Văn Liêm     & 12 & $8.0$ & $8.1$ & $0.1$ \\
	S04 & Thực hành Sư phạm & 11 & $8.8$ & $8.7$ & $0.1$ \\
	S05 & Thực hành Sư phạm & 12 & $9.4$ & $9.6$ & $0.2$ \\
	S06 & Thực hành Sư phạm & 11 & $9.4$ & $9.6$ & $0.2$ \\
	S07 & Thực hành Sư phạm & 11 & $7.8$ & $8.0$ & $0.2$ \\
	S08 & Lý Tự Trọng       & 12 & $9.5$ & $9.1$ & $0.4$ \\
	S09 & Thực hành Sư phạm & 11 & $8.9$ & $8.5$ & $0.4$ \\
	S10 & Thực hành Sư phạm & 11 & $6.1$ & $6.5$ & $0.4$ \\
	S11 & Thực hành Sư phạm & 11 & $9.1$ & $9.6$ & $0.5$ \\
	S12 & Lý Tự Trọng       & 12 & $8.9$ & $9.4$ & $0.5$ \\
	S13 & Thực hành Sư phạm & 11 & $9.5$ & $9.0$ & $0.5$ \\
	S14 & Châu Văn Liêm     & 12 & $6.9$ & $7.4$ & $0.5$ \\
	S15 & Châu Văn Liêm     & 12 & $8.7$ & $8.0$ & $0.7$ \\
	S16 & Thực hành Sư phạm & 11 & $9.8$ & $9.1$ & $0.7$ \\
	S17 & Thực hành Sư phạm & 11 & $8.8$ & $8.0$ & $0.8$ \\
	S18 & Thực hành Sư phạm & 11 & $7.9$ & $9.0$ & $1.1$ \\
	S19 & Thực hành Sư phạm & 11 & $6.9$ & $8.0$ & $1.1$ \\
	S20 & Lý Tự Trọng       & 12 & $7.2$ & $8.5$ & $1.3$ \\
	S21 & Thực hành Sư phạm & 11 & $9.8$ & $8.5$ & $1.3$ \\
	S22 & Châu Văn Liêm     & 12 & $7.0$ & $8.4$ & $1.4$ \\
	S23 & Thực hành Sư phạm & 11 & $6.7$ & $8.2$ & $1.5$ \\
	S24 & Thực hành Sư phạm & 11 & $6.9$ & $8.5$ & $1.6$ \\
	S25 & Thực hành Sư phạm & 11 & $6.9$ & $8.5$ & $1.6$ \\
	S26 & Thực hành Sư phạm & 11 & $5.0$ & $6.7$ & $1.7$ \\
	S27 & Thực hành Sư phạm & 11 & $5.3$ & $7.2$ & $1.9$ \\
	S28 & Châu Văn Liêm     & 12 & $6.5$ & $8.6$ & $2.1$ \\
	S29 & Thực hành Sư phạm & 12 & $5.0$ & $7.1$ & $2.1$ \\
	S30 & Thực hành Sư phạm & 11 & $2.9$ & $5.0$ & $2.1$ \\
	S31 & Thực hành Sư phạm & 11 & $6.9$ & $9.1$ & $2.2$ \\
	S32 & Lý Tự Trọng       & 11 & $6.9$ & $9.1$ & $2.2$ \\
	S33 & Thực hành Sư phạm & 11 & $6.9$ & $9.2$ & $2.3$ \\
	S34 & Lý Tự Trọng       & 12 & $6.2$ & $8.5$ & $2.3$ \\
	S35 & Thực hành Sư phạm & 11 & $7.0$ & $9.4$ & $2.4$ \\
	S36 & Thực hành Sư phạm & 11 & $6.0$ & $8.6$ & $2.6$ \\
	S37 & Thực hành Sư phạm & 11 & $6.2$ & $9.0$ & $2.8$ \\
	S38 & Châu Văn Liêm     & 12 & $6.6$ & $9.5$ & $2.9$ \\
\end{longtable}\par

Các kết quả đánh giá được thống kê lại theo các mốc: \begin{itemize}
	\item \textit{Giỏi}: Từ $8.0$ trở lên.
	\item \textit{Khá}: Từ $6.5$ tới dưới $7.9$.
	\item \textit{Trung bình}: Từ $5.0$ tới dưới $6.4$.
	\item \textit{Yếu}: Từ $3.5$ tới dưới $4.9$.
	\item \textit{Kém}: Dưới $3.5$.
\end{itemize}\par

\begin{longtable}{SlScScScSc}
	\multirow{2}{*}{\bfseries Nhóm} & \multicolumn{2}{Sc}{\bfseries KQ đánh giá} & \multicolumn{2}{Sc}{\bfseries ĐTB Toán}\\\cline{2-5}
	& Tần số & Tỉ lệ & Tần số & Tỉ lệ \\\hline\endhead\hline\endfoot
	Giỏi       & $15$ & $39.5\%$ & $32$ & $84.2\%$ \\
	Khá        & $15$ & $39.5\%$ &  $5$ & $13.2\%$ \\
	Trung bình &  $7$ & $18.4\%$ &  $1$ &  $2.6\%$ \\
	Yếu        &  $-$ & $-$ & $-$ & $-$ \\
	Kém        &  $1$ &  $2.6\%$ &  $-$ & $-$ \\
\end{longtable}\par

Đề tài tiến hành kiểm định tính chính xác của kết quả đánh giá, tức là kiểm định tính độc lập giữa hai biến \textit{KQ đánh giá} và \textit{ĐTB Toán}. Ở đây, mỗi học sinh được phân loại theo hai đặc tính là \textit{KQ đánh giá} (đặt là biến $X$) và \textit{ĐTB Toán} (đặt là biến $Y$). Có 4 giá trị cho biến $X$ (giỏi, khá, trung bình, kém) và 3 giá trị cho $Y$ (giỏi, khá, trung bình) nên $$P_{ij}=P(X=x_i,Y=y_j),~i=\overline{1,4},~j=\overline{1,3}.$$
là xác suất chọn được học sinh mạng đặc tính $X$ là $x_i$ và đặc tính $Y$ là $y_j$. Gọi
$$p_i=P(X=x_i)=\sum_{j=1}^{3}P_{ij},~i=\overline{1,4},$$
$$q_j=P(Y=y_j)=\sum_{i=1}^{4}P_{ij},~j=\overline{1,3}.$$
Trong đó $p_i$ là xác suất chọn được học sinh có $X=x_i$, $q_j$ là xác suất chọn được học sinh có $Y=y_j$.\par
Phát biểu giả thuyết:
\begin{align*}
	H_0:~&P_{ij}=p_iq_j,~\forall i=\overline{1,4},~j=\overline{1,3}.\\
	H_1:~&\exists (i,j) | P_{ij}\neq p_iq_j.
\end{align*}\par

Từ kết quả thực nghiệm đánh giá (bảng \ref{tab:tab-s4-result}), thu được bảng kết quả như sau:
\begin{longtable}{ScSlScScScSc}
	& & \multicolumn{3}{Sc}{\bfseries ĐTB Toán} & \multirow{2}{*}{\bfseries Tổng}\\\cline{3-5}
	& & Giỏi & Khá & Trung bình &\\\hline\endhead\hline\endfoot
	\multirow{4}{*}{\textbf{KQ đánh giá}}
	& Giỏi       & 15 & – & – & 15\\
	& Khá        & 14 & 1 & – & 15\\
	& Trung bình & 3  & 4 & – & 7 \\
	& Kém        & 0  & 0 & 1 & 1 \\\hline
	\multicolumn{2}{Sc}{\textbf{Tổng}} & 32 & 5 & 1 & 38
\end{longtable}\par

Kết quả kiếm định độc lập bằng phần mềm SPSS được cho trong bảng \ref{tab:tab-s4-chi-square} dưới đây:
\begin{longtable}{SlSrScSc}
	\caption{Kết quả kiểm định ($\chi^2$) \textit{KQ đánh giá} với \textit{ĐTB Toán}}\label{tab:tab-s4-chi-square}\\
	& \multicolumn{1}{Sc}{\bfseries Value} & \textbf{df} & \textbf{A. Sign. (2-sided)}\\\hline\endfirsthead

	& \multicolumn{1}{Sc}{\bfseries Value} & \textbf{df} & \textbf{A. Sign. (2-sided)}\\\hline\endhead\hline\endfoot
	\textbf{Pearson Chi-Square} & $52.734$ & $6$ & $<0.001$\\
	\textbf{Likelihood Ratio} & $21.646$ & $6$ & $0.001$\\
	\textbf{N of Valid Cases} & 38 & &\\
\end{longtable}
Kết quả $p\mathrm{-value}=0.001<5\%$ nên có thể bác bỏ giả thuyết $H_0$ tại mức ý nghĩa $5\%$ và kết luận rằng \textit{KQ đánh giá} phụ thuộc vào \textit{ĐTB Toán}. Hơn nữa, điều này khẳng định việc đánh giá năng lực học sinh thông qua Kant bot là hoàn toàn khả quan.\par

Tiếp đến, đề tài tiến hành ước lượng sai số trong đánh giá, kết quả được cho ở bảng \ref{tab:tab-s4-explore}.
\begin{longtable}{SlSlScSc}
	\caption{Kết quả ước lượng \textit{Sai số} đánh giá}\label{tab:tab-s4-explore}\\
	& & \textbf{Statistic} & \textbf{Std. Error} \\\hline\endfirsthead

	& & \textbf{Statistic} & \textbf{Std. Error} \\\hline\endhead\hline\endfoot

	\multicolumn{2}{Sl}{Mean} & $1.232$ & $0.1449$\\
	\multirow{2}{*}{\begin{minipage}{3.5cm}Confidence Interval for Mean\end{minipage}}
	& Lower Bound & $0.987$&\\
	& Upper Bound & $1.476$  &\\
	\multicolumn{2}{Sl}{5\% Trimmed Mean}    & $1.207$  &\\
	\multicolumn{2}{Sl}{Median}              & $1.2$    &\\
	\multicolumn{2}{Sl}{Variance}            & $0.797$  &\\
	\multicolumn{2}{Sl}{Std. Deviation}      & $0.8929$ &\\
	\multicolumn{2}{Sl}{Minimun}             & $0.0$    &\\
	\multicolumn{2}{Sl}{Maximum}             & $2.9$    &\\
	\multicolumn{2}{Sl}{Range}               & $2.9$    &\\
	\multicolumn{2}{Sl}{Interquartile Range} & $1.7$    &\\
	\multicolumn{2}{Sl}{Skewness}            & $0.249$  & $0.383$\\
	\multicolumn{2}{Sl}{Kurtosis}            & $-1.313$ & $0.750$\\
\end{longtable}\par

Với độ tin cậy $90\%$, \textit{Sai số} đánh giá so với \textit{ĐTB Toán} của học sinh được ước lượng nằm trong đoạn $[0.987;1.476]$ nhìn chung khá lớn, tuy nhiên đây chỉ là bài kiểm tra năng lực của một chủ đề (phần Tổ hợp, xác suất), do đó hoàn toàn có thể chấp nhận được. Bên cạnh đó, \textit{độ lệch chuẩn} (standard deviation) của \textit{Sai số} là $0.8929<1$ – hoàn toàn tương ứng với điều kiện dừng của thuật toán (dừng kiểm tra khi kỳ vọng nhỏ hơn $1$).\par
