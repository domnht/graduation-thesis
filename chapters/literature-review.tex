\chapter{Cơ sở lý luận}

\addcontentsline{toc}{chapter}{Mở đầu}
\section{Lorem ipsum dolor sit amet}
Lorem ipsum dolor sit amet, consectetur adipiscing elit. Vivamus iaculis quis quam a mollis. Maecenas vulputate viverra dui, vitae luctus elit porta sed. In fringilla eu augue ac pharetra. Donec sodales sem tortor, luctus fermentum est semper eu. Morbi ac leo vel sapien interdum commodo. Vivamus non aliquam leo. Aliquam ac enim et sapien imperdiet mollis pretium id nisi. Mauris non sagittis tortor. Vestibulum commodo, ante vel bibendum pharetra, neque enim venenatis tellus, sed dictum lectus justo vel diam.\par
\subsection{Nulla facilisi}
Nulla facilisi. Etiam quis sapien vel dui tempus volutpat. Cras ut turpis non turpis posuere volutpat luctus eu magna. Curabitur ornare tellus felis, non hendrerit nisi luctus tristique. Aenean mollis faucibus scelerisque. Aenean commodo feugiat quam, hendrerit fringilla arcu feugiat in. Aliquam cursus luctus ex. Aenean aliquet varius nibh sit amet ullamcorper. Lorem ipsum dolor sit amet, consectetur adipiscing elit. Pellentesque posuere leo ac vestibulum laoreet. Sed molestie, tellus vel vulputate malesuada, nulla leo imperdiet turpis, in facilisis libero mauris at odio.\par
\section{Integer tincidunt sagittis turpis}
Integer tincidunt sagittis turpis, at tincidunt sapien viverra ut. Fusce maximus est et nulla consectetur euismod. Maecenas accumsan vestibulum vehicula. Cras molestie odio ac ex tincidunt, in ullamcorper lectus tristique. Curabitur luctus sagittis arcu eget viverra. Curabitur eget justo odio. Duis vel neque sollicitudin, eleifend elit sit amet, facilisis lacus. Nam tempor elementum convallis. Aliquam id magna sed purus porta mollis vitae quis quam. Aenean ac dolor euismod, pellentesque quam eget, efficitur magna.\par

Using \texttt{biblatex} you can display bibliography divided into sections, 
depending of citation type. 
Let's cite! Einstein's journal paper \cite{bai-bao-1} and the Dirac's 
book \cite{sach-1} are physics related items. 
Next, \textit{The \LaTeX\ Companion} book \cite{website-1}, the Donald Knuth's website \cite{sach-2}; but the others Donald Knuth's items \cite{bai-bao-1} are dedicated to programming.