\chapter{Cơ sở lý luận}
% \addcontentsline{toc}{chapter}{Mở đầu}

\section{Trí tuệ nhân tạo (AI) và học máy (ML)}
(\cite{nguyen2018tri})


\section{Trí tuệ nhân tạo trong giáo dục}
(\cite{garito1991artificial})
(\cite{beck1996applications})
(\cite{goksel2019artificial})
(\cite{devedvzic2004web})

\section{Nến tảng Chatbot}
\subsection{Khái quát về Chatbot}
Chatbot là gì?\par
Tại sao Chatbot được xem là AI?\par
(\cite{bii2013chatbot})
\subsection{Ứng dụng của Chatbot trong giáo dục}
(\cite{10.1007/978-3-030-01689-0_23})
(\cite{hoang2011ung})
(\cite{hsu2012mobile})

\section{Các phương pháp đánh giá kết quả học tập Toán học}

% Using \texttt{biblatex} you can display bibliography divided into sections, 
% depending of citation type. 
% Let's cite! Einstein's journal paper \cite{bai-bao-1} and the Dirac's 
% book \cite{sach-1} are physics related items. 
% Next, \textit{The \LaTeX\ Companion} book \cite{website-1}, the Donald Knuth's website \cite{sach-2}; but the others Donald Knuth's items \cite{bai-bao-1} are dedicated to programming.
