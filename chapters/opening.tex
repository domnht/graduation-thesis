\chapter*{Mở đầu}
\addcontentsline{toc}{chapter}{Mở đầu}

% Set item list 1 - a
\renewcommand*{\thesection}{\arabic{section}}
\renewcommand*{\thesubsection}{\alph{subsection}}

\section{Lý do chọn đề tài}
Cuộc cách mạng công nghiệp lần thứ tư hay còn gọi là cuộc cách mạng số diễn ra từ đầu thế kỷ XXI. Đặc trưng của cuộc cách mạng công nghiệp lần này là sẽ ngày càng phổ biến trí thông minh nhân tạo và máy móc tự động hóa, đem lại sự kết hợp giữa hệ thống ảo và thực tế. Cuộc cách mạng này tác động mạnh mẽ đến nhiều lĩnh vực, nhiều khía cạnh trong đời sống xã hội, trong đó đặc biệt không thể thiếu một nguồn nhân lực chất lượng cao; mà nguồn nhân lực lại là đối tượng trực tiếp của giáo dục – đào tạo.\par
Tuy nhiên, lĩnh vực AI ở Việt Nam chỉ mới được nghiên cứu và ứng dụng trên một số lĩnh vực như thông tin – truyền thông, y tế, du lịch... với một tỉ trọng rất nhỏ, thậm chí còn chưa được các doanh nghiệp quan tâm. Trong đó, việc đưa AI vào lĩnh vực giáo dục vẫn là một bài toán khó, thể hiện qua số lượng hạn chế các công trình nghiên cứu về AI trong giáo dục – đa số là nghiên cứu lý luận – và chưa có nhiều sản phẩm dành cho giáo dục.\par
Bên cạnh đó, các nền tảng mạng xã hội ngày càng phát triển và phổ biến, đặc biệt là với học sinh phổ thông. Đây là một môi trường tốt để hỗ trợ quá trình tự học của học sinh, cụ thể là cung cấp tài liệu bổ sung, bài tập về nhà... Song vẫn chưa được khai thác triệt để và đúng mức.\par
Trong chương trình Toán phổ thông, chương Xác suất, thống kê (Toán 11) có nhiều ứng dụng quan trọng và có nhiều mảng kiến thức mang tính hàn lâm, khó nắm bắt. Trong khi năng lực tiếp cận Toán học ở các học sinh thường không đồng đều nhau, đòi hỏi phương pháp tiếp cận khác nhau ở từng em. Tuy nhiên, trong các lớp học thực tế, một giáo viên thường phải quản lý 30 – 40 học sinh, tạo ra sự bất khả thi trong việc nắm bắt kịp thời mức độ tiếp nhận kiến thức của từng em.\par
Xuất phát từ những lý do trên, tôi chọn đề tài nghiên cứu \textit{"Bước đầu ứng dụng trí tuệ nhân tạo vào dạy học Toán: Một thực nghiệm của Facebook Chatbot trong dạy học Toán 11 chương Tổ hợp, xác suất"}.\par

\section{Mục tiêu nghiên cứu}
Mục tiêu của luận văn này là xây dựng một \textit{máy trò chuyện} (chatbot) trên nền tảng mạng xã hội Facebook, trong đó cung cấp các câu hỏi trắc nghiệm, được tự động phân bố theo năng lực học tập của học sinh, sử dụng vào phần bài tập về nhà.

\section{Nhiệm vụ nghiên cứu}
Luận văn thực hiện những nhiệm vụ sau:\par
\begin{enumerate}[label=\textbf{\thesection.\arabic*.},align=left,left=0cm..1cm]
	\item Tìm hiểu vai trò và các ứng dụng của AI trong dạy học Toán học.
	\item Xây dựng một giao thức (API) xử lý thông tin với ngôn ngữ PHP và các thư viện, thuật toán học máy (Machine Learning – ML).
	\item Thiết kế một AI Chatbot trên nền tảng Facebook.
	\item Vận dụng AI Chatbot vào bài tập về nhà.
	\item Tiến hành thực nghiệm sư phạm để đánh giá tính khả thi và xác định ưu nhược điểm khi sử dụng AI Chatbot trong dạy học.
\end{enumerate}\par

\section{Đối tượng nghiên cứu}
\begin{enumerate}[label=\textbf{\thesection.\arabic*.},align=left,left=0cm..1cm]
\item Ứng dụng của AI trong giáo dục.\par
\item Hoạt động dạy và học của giáo viên và học sinh.\par
\end{enumerate}

\section{Phạm vi nghiên cứu}
\begin{enumerate}[label=\textbf{\thesection.\arabic*.},align=left,left=0cm..1cm]
	\item \textbf{Về phương pháp:} giới hạn sử dụng nền tảng Chatbot của mạng xã hội Facebook, ngôn ngữ lập trình PHP và một số thuật toán ML.\par
	\item \textbf{Về chuyên môn:} giới hạn trong chương trình Toán 11 cơ bản chương Xác suất, thống kê.\par
\end{enumerate}

\section{Phương pháp nghiên cứu}
\begin{enumerate}[label=\textbf{\thesection.\arabic*.},align=left,left=0cm..1cm]
	\item \textbf{Phương pháp nghiên cứu lý luận}\par
	Nghiên cứu các tài liệu về triết học, tâm lý học, giáo dục học lý luận dạy học, các phương pháp và ứng dụng công nghệ trong giáo dục nói chung và trí tuệ nhân tạo nói riêng.
	\item \textbf{Phương pháp thực nghiệm} \par
	Từ các nghiên cứu lý luận, sử dụng các công cụ để thiết kế Chatbot trên nền tảng Facebook.
	\item \textbf{Phương pháp điều tra, quan sát} \par
	Tiến hành dạ thực nghiệm và thu thập thông tin từ phiếu khảo sát về mức độ hứng thú của học sinh qua bài học.
	\item \textbf{Phương pháp thống kê Toán học} \par
	Phân tích định tính, định lượng, từ đó rút ra kết luận về tính khả thi cũng như ưu/nhược điểm của nền tảng Chatbot.
\end{enumerate}

% Reset item list to x.1 and x.1.1
\renewcommand*{\thesection}{\arabic{chapter}.\arabic{section}}
\renewcommand*{\thesubsection}{\arabic{chapter}.\arabic{section}.\arabic{subsection}}
