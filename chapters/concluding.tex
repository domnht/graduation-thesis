\chapter*{Kết luận}
\addcontentsline{toc}{chapter}{Kết luận}

\section*{Đóng góp của luận văn}
Trong quá trình nghiên cứu và xây dựng chatbot hỗ trợ đánh giá năng lực học sinh, luận văn đã đạt được một số kết quả sau: \begin{itemize}
	\item Luận văn đã tìm hiểu cơ bản lý thuyết ứng đáp câu hỏi và dựa vào đó để xây dựng thuật toán trắc nghiệm thích ứng.
	\item Tìm hiểu khả năng và cách sử dụng nền tảng Chatfuel trong thiết kế Facebook chatbot.
	\item Mã hóa lý thuyết ứng đáp câu hỏi và thuật toán trắc nghiệm thích ứng thành một class bằng ngôn ngữ lập trình PHP.
	\item Xây dựng một JSON API xử lý và thiết kế Kant bot, ứng dụng vào đánh giá năng lực học sinh.
\end{itemize}

\section*{Hạn chế của luận văn}
Trong quá trình hoàn thành luận văn và chatbot, mặc dù đã đạt được một số kết quả nhất định trong đánh giá năng lực học sinh, vẫn có những hạn chế nhất định: \begin{itemize}
	\item Ngân hàng câu hỏi còn hạn chế, chưa thật sự bao quát tất cả mức năng lực, dẫn tới việc chọn câu hỏi có độ chênh lệch với mức năng lực khá cao.
	\item Bot chỉ đưa ra kết quả đúng/sai cho mỗi câu hỏi, và kết quả đánh giá vào cuối quá trình, chưa gợi ý được các kiến thức liên quan cần ôn luyện để cải thiện kết quả.
	\item Văn phong hội thoại của chatbot chưa đa dạng, chưa thu hút được học sinh.
\end{itemize}

\section*{Hướng phát triển}
Với sự phức tạp của đánh giá, luận văn còn nhiều hướng có thể phát triển tiếp tục: \begin{itemize}
	\item Mở rộng ngân hàng câu hỏi sao cho đủ bao quát được tất cả mức năng lực.
	\item Tối ưu phần xử lý, bao gồm các class và JSON API.
	\item Đa dạng hóa văn phong hội thoại của chatbot.
\end{itemize}
