\chapter*{Đề cương luận văn}
\addcontentsline{toc}{chapter}{Đề cương luận văn}

Đề tài: \textit{\textbf{Bước đầu ứng dụng trí tuệ nhân tạo vào dạy học Toán: Một thực nghiệm của Facebook Chatbot trong dạy học Toán 11 chương Tổ hợp, xác suất}}\par
Sinh viên: \textbf{Nguyễn Hiếu Thanh}\par
% Set item list 1 - a
\renewcommand*{\thesection}{\arabic{section}}
\renewcommand*{\thesubsection}{\alph{subsection}}

\section{Lý do chọn đề tài}
Việt Nam đang từng bước hội nhập vào cuộc cách mạng Công nghiệp 4.0, đây thời kỳ phát triển của Trí tuệ nhân tạo (AI).\par
Các ứng dụng công nghệ thông tin ngày càng được sử dụng rãi trên nhiều lĩnh vực, trong đó có giáo dục. Tuy nhiên, đa số công cụ mang tính cứng nhắc, không mang tính cá nhân hóa với người dùng do chưa được tích hợp thuật toán AI thích hợp.\par
Trí tuệ nhân tạo (AI) trong giáo dục chưa được nghiên cứu rộng rãi, và/hoặc chỉ ở mức nghiên cứu lý thuyết.\par
Các nền tảng mạng xã hội ngày càng phát triển và được sử dụng rộng rãi, đặc biệt là với học sinh phổ thông, nhưng chưa được khai thác triệt để cho các ứng dụng giáo dục.\par
Chương Xác suất, thống kê trong chương trình Toán phổ thông có nhiều ứng dụng quan trọng và có nhiều mảng kiến thức khó nắm bắt.\par
Xuất phát từ những lý do trên, tôi chọn đề tài nghiên cứu \textit{"Bước đầu ứng dụng trí tuệ nhân tạo vào dạy học Toán: Một thực nghiệm của Facebook Chatbot trong dạy học Toán 11 chương Tổ hợp, xác suất"}.\par
Trích dẫn: \cite{goksel2019artificial}, \cite{garito1991artificial}, \cite{beck1996applications}, \cite{jones1985applications}, \cite{nguyen2018tri}, \cite{timms2016letting}\par

\section{Mục tiêu nghiên cứu}
Nghiên cứu của luận văn này là xây dựng một \textit{hộp trò chuyện} (chatbot) trên nền tảng mạng xã hội Facebook, trong đó cung cấp các câu hỏi trắc nghiệm, được tự động phân bố theo năng lực học tập của học sinh, sử dụng vào phần củng cố và bài tập về nhà.

\section{Nhiệm vụ nghiên cứu}
Luận văn thực hiện những nhiệm vụ sau:\par
\begin{enumerate}[label=\textbf{\thesection.\arabic*.},align=left,left=0cm..1cm]
	\item Tìm hiểu về vai trò và các ứng dụng của AI trong dạy học Toán học.
	\item Xây dựng một giao thức (API) xử lý thông tin với ngôn ngữ PHP.
	\item Thiết kế một AI Chatbot trên nền tảng Facebook.
	\item Vận dụng AI Chatbot vào phần củng cố trong bài dạy.
	\item Vận dụng AI Chatbot vào bài tập về nhà.
	\item Tiến hành thực nghiệm sư phạm để đánh giá tính khả thi và xác định ưu nhược điểm khi sử dụng AI Chatbot trong dạy học.
\end{enumerate}\par
Trích dẫn: \cite{gadanidis2017artificial}, \cite{sheromova2020learning}, \cite{themistokleous2020mobile}, \cite{martin2017virtual}, \cite{bii2013chatbot}, \cite{hoang2011ung}, \cite{hsu2012mobile}, \cite{abueloun2017mathematics} \cite{devedvzic2004web}\par

\section{Đối tượng nghiên cứu}
Ứng dụng của AI trong giáo dục.\par
Hoạt động dạy và học của giáo viên và học sinh.\par

\section{Phạm vi nghiên cứu}
Về phương pháp: giới hạn sử nền tảng Chatbot của mạng xã hội Facebook.\par
Về chuyên môn: giới hạn trong chương trình Toán 11 cơ bản chương Xác suất, thống kê.\par

\section{Phương pháp nghiên cứu}
\begin{enumerate}[label=\textbf{\thesection.\arabic*.},align=left,left=0cm..1cm]
	\item \textbf{Phương pháp nghiên cứu lý luận}\par
	Nghiên cứu các tài liệu về triết học, tâm lý học, giáo dục học lý luận dạy học, các phương pháp và ứng dụng công nghệ trong giáo dục nói chung và trí tuệ nhân tạo nói riêng.
	\item \textbf{Phương pháp thực nghiệm} \par
	Từ các nghiên cứu lý luận, sử dụng các công cụ để thiết kế Chatbot trên nền tảng Facebook.
	\item \textbf{Phương pháp điều tra, quan sát} \par
	Tiến hành dạ thực nghiệm và thu thập thông tin từ phiếu khảo sát về mức độ hứng thú của học sinh qua bài học.
	\item \textbf{Phương pháp thống kê Toán học} \par
	Phân tích định tính, định lượng, từ đó rút ra kết luận về tính khả thi cũng như ưu/nhược điểm của nền tảng Chatbot.
\end{enumerate}
\section{Kế hoạch nghiên cứu}
\begin{tabular}{|p{.1\linewidth}|p{.35\linewidth}|p{.2\linewidth}|p{.25\linewidth}|}
	\hline
	\textbf{TG} & \textbf{Nội dung} & \textbf{Phương pháp nghiên cứu} & \textbf{Kết quả} \\
	\hline
	10 – 11/2020 & Nghiên cứu cơ sở lý luận; Tìm hiểu về lĩnh vực trí tuệ nhân tạo nói chung và Chatbot nói riêng & Phân tích, tổng hợp & Cơ sở lý thuyết của luận văn \\
	\hline
	12 - 01/2020 & Thiết kế API và Chatbot & Nghiên cứu và phát triển & Chatbot trên nền tảng Facebook \\
	\hline
	02/2020 & Thực nghiệm Sư phạm & Thực nghiệm, điều tra, thống kê toán học & Kiểm chứng tính khả thi và hiệu quả của AI Chatbot trong dạy học Toán \\
	\hline
	02 – 04/2020 & Viết luận văn & Tổng hợp & Hoàn chỉnh luận văn \\
	\hline
	05/2020 & Báo cáo luận văn & & \\
	\hline
\end{tabular}
\section{Cấu trúc luận văn}
Luận văn gồm có 03 phần chính:
\begin{enumerate}[label=\textbf{Phần \arabic*.},align=left,left=0cm]
	\item Mở đầu
	\item Nội dung: Gồm 04 chương:\par
		\begin{enumerate}[label=Chương \arabic*.,align=left]
			\item Cơ sở lý luận
			\item Xây dựng API xử lý
			\item Thiết kế AI Chatbot
			\item Thực nghiệm sư phạm
		\end{enumerate}
	\item Kết luận
\end{enumerate}

% Reset item list to x.1 and x.1.1
\renewcommand*{\thesection}{\arabic{chapter}.\arabic{section}}
\renewcommand*{\thesubsection}{\arabic{chapter}.\arabic{section}.\arabic{subsection}}
